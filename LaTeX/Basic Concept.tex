\documentclass[lang = cn]{elegantpaper}
\usepackage{amsmath,bm}
\usepackage{amsfonts,amssymb}
\usepackage{graphicx}
\usepackage{pythonhighlight}
\usepackage {xeCJK}
\title{基本概念及理解}
\author{Atri}
\begin{document}
\maketitle
无论是听课还是自学, 我发现大部分同学都存在一个问题, 就是对某些基础概念相当陌生, 这对听课效率乃至后面自学都产生了较大的影响. 因此, 我打算专门做个 \LaTeX 的 note 来总结一下.
\section{超平面(hyperplane)}
\noindent
课件所给的超平面的形式:
\begin{equation*}
    L = \left\{\bm{x}\in\mathbb{R}^k|A\bm{x}=\bm{b},A\in\mathbb{R}^{d \times k}\right\}.
\end{equation*}
显然, 我完全不知道为什么长这样. 问了 TA 表示也不知道. 那我们就暂且先不管这个定义.\\
给出维基百科对hyperplane的定义: In geometry a hyperplane is a subspace of one dimension less than its ambient space. 也就是说, 超平面是比当前空间少一个维度的子空间, 这个子空间把该空间分成了两个部分.\\
设 $\bm{x}_0$ 是超平面上的点, $\bm{\omega}$ 为超平面的法向量. 根据法向量正交于任何超平面上的向量的性质, 可以得出, 对超平面上任意一点 $\bm{x}$, 有:
\begin{equation*}
    \bm{\omega}^\top(\bm{x}-\bm{x}_0) = 0.
\end{equation*}
即
\begin{equation*}
    \bm{\omega}^\top \bm{x} = \bm{\omega}^\top \bm{x}_0 = \bm{b}.
\end{equation*}
因此可以得出超平面的一般形式:
\begin{equation*}
    \bm{\omega}^\top \bm{x} = \bm{b}.
\end{equation*}
可以认为, $\bm{\omega}$ 决定超平面的姿态, 而 $\bm{b}$ 决定超平面的位置.
\section{投影}
\noindent
对于空间内的一条直线, 可以使用点向式定义:
\begin{equation*}
    S = \left\{\bm{y}|\bm{y} = \bm{p} + t\bm{v}, t\in\mathbb{R}\right\}.
\end{equation*}
这里主要研究点到直线的投影.\\
根据生活常识, 投影的定义是直线上与空间中的点相对应的一个点, 二者满足距离最小, 即
\begin{equation*}
    \min_{t \in \mathbb{R}} \|\bm{x}-(\bm{p}+t\bm{v})\|^2 = \|\bm{x}-\bm{p}\|^2 +t^2 \|\bm{v}\|^2 - 2t\bm{v}^\top(\bm{x}-\bm{p}).
\end{equation*}
为了方便, 令方向向量 $\bm{v}$ 为单位向量, 因此我们只需求出放缩系数 $t$ :
\begin{equation*}
    t = \arg \min_{t\in\mathbb{R}} t^2 -2t\bm{v}^\top(\bm{x}-\bm{p}) + \|\bm{x}-\bm{p}\|^2.
\end{equation*}
这是关于 $t$ 的二次函数, 极小值点 $t^{*} = \bm{v}^\top(\bm{x}-\bm{p})$, 代入可得投影点:
\begin{equation*}
    \widehat{\bm{x}} = \bm{p} + (\bm{v}^\top(\bm{x}-\bm{p}))\bm{v}.
\end{equation*}
若直线过原点, 可得
\begin{equation*}
    \widehat{\bm{x}} = (\bm{v}^\top \bm{x})\bm{v}.
\end{equation*}
\section{协方差矩阵}
\noindent
在概率论中, 随机变量 $\mathbf{X}$ 和 $\mathbf{Y}$ 的协方差为:
\begin{equation*}
    \mathrm{Cov}(\mathbf{X},\mathbf{Y}) = \mathbb{E}\left\{[\mathbf{X}-\mathbb{E}(\mathbf{X})][\mathbf{Y}-\mathbb{E}(\mathbf{Y})]\right\}.
\end{equation*}
拓展到高维的情况, 其协方差矩阵 $C$ 的第 $i$ 行第 $j$ 列为: $\mathrm{Cov}(\mathbf{X}_i,\mathbf{X}_j)$. 显然协方差矩阵是一个对称矩阵.\\
然而, 在概率论中, 我们对样本协方差的定义是一个统计量, 有以下两种形式(假设样本均值为 0):
\begin{equation*}
    S = \frac{1}{N}\sum_{i=1}^N \bm{x}_i\bm{x}_i^\top \quad \mathrm{or} \quad S = \frac{1}{N-1}\sum_{i=1}^N \bm{x}_i\bm{x}_i^\top.
\end{equation*}
前者是课件中所用的形式, 在概率论中可作为协方差的最大似然估计;相对应的, 后者是协方差的一种无偏估计.\\
其中, 以下秩一矩阵的和, 被称为散布矩阵, 它是个半正定矩阵:
\begin{equation*}
    \sum_{i=1}^N \bm{x}_i\bm{x}_i^\top.
\end{equation*}
在FLDA(线性判别分析)的推导中也有用到相关概念: https://zhuanlan.zhihu.com/p/625837046.
\section{矩阵求导的本质}
\noindent
高等数学的时候我们学过偏导, 它是将一个 function 对所有自变量分别求导.\\
矩阵求导也是一样的, 本质就是 function 中的每个 $f$ 分别对变元中的每个元素逐个求偏导, 只不过写成了向量、矩阵的形式而已.
\subsection{向量变元的实值标量函数布局}
\noindent
直观上来看, \textbf{分子布局}, 就是分子是列向量的形式, 分母是行向量的形式:
\begin{equation*}
    \dfrac{\partial \bm{f}_{2 \times 1}(\bm{x})}{\partial \bm{x}_{3 \times 1}^\top} = \begin{bmatrix}
        \frac{\partial f_1}{\partial x_1} & \frac{\partial f_1}{\partial x_2} & \frac{\partial f_1}{\partial x_3} \\
        \frac{\partial f_2}{\partial x_1} & \frac{\partial f_2}{\partial x_2} & \frac{\partial f_2}{\partial x_3}
    \end{bmatrix}_{2 \times 3}
\end{equation*}
而\textbf{分母布局}, 就是分子是行向量的形式, 分母是列向量的形式:
\begin{equation*}
    \dfrac{\partial \bm{f}_{2 \times 1}^\top(\bm{x})}{\partial \bm{x}_{3 \times 1}} = \begin{bmatrix}
        \frac{\partial f_1}{\partial x_1} & \frac{\partial f_2}{\partial x_1} \\
        \frac{\partial f_1}{\partial x_2} & \frac{\partial f_2}{\partial x_2} \\\frac{\partial f_1}{\partial x_3} & \frac{\partial f_2}{\partial x_3}
    \end{bmatrix}_{3 \times 2}
\end{equation*}
\subsection{矩阵变元的实值标量函数布局}
\noindent
这里只介绍矩阵的形式.\\
Jocabian矩阵形式, 即先把矩阵变元 $\bm{X}$ 进行转置, 然后逐分量求导:
\begin{equation*}
    \dfrac{\partial f(\bm{X})}{\partial \bm{X}_{m \times n}^\top} = \begin{bmatrix}
        \frac{\partial f}{\partial x_{11}} & \frac{\partial f}{\partial x_{21}} & \cdots & \frac{\partial f}{\partial x_{m1}}\\
        \frac{\partial f}{\partial x_{12}} & \frac{\partial f}{\partial x_{22}} & \cdots & \frac{\partial f}{\partial x_{m2}}\\
        \vdots & \vdots & \vdots & \vdots \\
        \frac{\partial f}{\partial x_{1n}} & \frac{\partial f}{\partial x_{2n}} & \cdots & \frac{\partial f}{\partial x_{mn}}
    \end{bmatrix}_{n \times m}
\end{equation*}
梯度向量形式, 即直接逐分量求导, 为上述形式的转置.
\section{链式法则}
\noindent
在\textbf{动手学深度学习}这门课中, 给出的链式法则公式如下:
\begin{equation*}
    \frac{\partial \bm{y}}{\partial \bm{x}} = \frac{\partial \bm{y}}{\partial \bm{u}} \frac{\partial \bm{u}}{\partial \bm{x}}.
\end{equation*}
然而, Lect3(or cmu-10315)给出的微分的链式法则: (令$\bm{u} = g(\bm{x})$)
\begin{equation*}
    \nabla f\circ g(\bm{x}) = \frac{\partial f\circ g(\bm{x})}{\partial \bm{x}} = \frac{\partial \bm{u}}{\partial \bm{x}} \frac{\partial f}{\partial \bm{u}}.
\end{equation*}
可以发现二者顺序相反. 个人认为应该是\textbf{动手学深度学习}这门课所选用的求导为分子布局, 而本课程用的是分母布局.\\
However, \textbf{动手学深度学习}所给链式法则更合乎常人理解, 其也是计算图上实现自动微分的原理.
\section{凸}
\noindent
如果一个优化问题可以被转化为一个凸优化问题, 那么这个问题就算解决了(
\subsection{凸集}
\begin{definition}
    集合 $C$ 被称为凸集, 当且仅当对任意的 $\bm{x},\bm{y} \in C$ 及 $0 \le \theta \le 1$, 都有
\begin{equation*}
    \theta \bm{x} + (1 - \theta)\bm{y} \in C.
\end{equation*}
\end{definition}
可以对比一下锥的概念: $\theta_1,\theta_2 > 0$, 不限制 $(\theta_1 + \theta_2 = 1)$, 都有
\begin{equation*}
    \theta_1 \bm{x} + \theta_2 \bm{y} \in C.
\end{equation*}
可以理解为两向量所\textbf{夹住}的一块锥形区域.\\
However, 在COP中用到的切锥的概念, 跟锥似乎一点关系没有?
\subsection{凸函数}
\noindent
主定义: $f(\lambda \bm{x} + (1 - \lambda)\bm{y}) \le \lambda f(\bm{x}) + (1 - \lambda) f(\bm{y})$, 严格凸对于 $\lambda \in (0,1)$ 不等式严格成立.\\
中点凸: $f((\bm{x}+\bm{y})/2) \le (f(\bm{x}) + f(\bm{y})) / 2$.\\
上面二者等价.\\
保凸运算: 设所有凸函数的集合为 $L$.\\
\begin{itemize}
    \item $f_1 \in L,f_2 \in L \Rightarrow f_1 + f_2 \in L$
    \item $f \in L,\alpha > 0 \Rightarrow \alpha f \in L$
    \item $f \in L \Rightarrow  f(A\bm{x} + b) \in L$.
\end{itemize}
\subsubsection{凸函数的判定}
\noindent
\begin{itemize}
    \item 定义.
    \item 一阶条件: $f(\bm{y}) \ge f(\bm{x}) + \nabla f(\bm{x})^\top(\bm{y}-\bm{x})$.
    \item 二阶条件: $\nabla^2 f(\bm{x}) \succeq 0$, 若正定为严格凸函数.
    \item 构造函数: $g = f(\bm{x} + t\bm{v})$ 为凸.
    \item 上方图为凸集.
\end{itemize}
\subsubsection{凸函数的性质}
\noindent
\begin{itemize}
    \item 所有的下水平集 $S_\alpha = \left\{\bm{x}\in\mathrm{dom}(f)|f(\bm{x}) \le \alpha\right\}$ 为凸集.
    \item 一、二阶条件.
    \item 上方图为凸集.
\end{itemize}
\subsubsection{强凸}
\noindent
$\mu$-强凸定义: $f(\bm{y}) \ge f(\bm{x}) + \nabla f(\bm{x})(\bm{y}-\bm{x}) + \frac{\mu}{2}\|\bm{y}-\bm{x}\|^2$.\\
另一个定义: $h = f(\bm{x}) - \frac{\mu}{2}\|\bm{x}\|^2$ 为凸函数.\\
一个性质: $(\nabla f(\bm{x}) - \nabla f(\bm{y}))^\top (\bm{x}-\bm{y}) \ge \mu \|\bm{x}-\bm{y}\|^2$.\\
对比一下 $L$-光滑函数:
\begin{equation*}
\begin{aligned}
    &f(\bm{y}) \le f(\bm{x}) + \nabla f(\bm{x})(\bm{y}-\bm{x}) + \frac{L}{2}\|\bm{y}-\bm{x}\|^2\\
    &h = \frac{L}{2}\|\bm{x}\|^2 - f(\bm{x}) \;\mathrm{is\;convex}\\
    &(\nabla f(\bm{x}) - \nabla f(\bm{y}))^\top (\bm{x}-\bm{y}) \le L \|\bm{x}-\bm{y}\|^2
\end{aligned}
\end{equation*}
\section{切空间}
\subsection{曲面}
\noindent
工科数学分析中, 我们在三维空间中常见的曲面有柱面、球面、旋转抛物面、双曲面等等. 一般来讲, 如果想用什么东西来表示一个曲面, 我们可能会去挖掘坐标 $(x,y,z)$ 的信息.\\
例如单位球面:
\begin{equation*}
    x^2 + y^2 + z^2 = 1.
\end{equation*}
然而, 事实上, 对于三维空间内的一条曲线, 也可以把它归结为曲面. 例如二次函数曲线:
\begin{equation*}
\begin{cases}
    y = x^2\\
    z = 0
\end{cases}
\end{equation*}
因此, 我们需要用维度来衡量一个曲面. 然而, 无论是上述的简单曲面和简单曲线, 都是处于 $\mathbb{R}^3$ 空间内的.\\
对于一个高维曲面, 我们或许可以用一种映射来表示:
\begin{equation*}
    M:\mathbb{R}^n \rightarrow \mathbb{R}^m,\bm{x}\mapsto F=(F_1(\bm{x}),\cdots,F_m(\bm{x})).
\end{equation*}
也就是说, 对于三维空间即 $m=3$, 我们或许可以用最少的相互独立的$n$个变量来确定空间, 每一个 $F_i(\bm{x})$ 或许代表一个坐标内的一个分量.\\
例如单位球面, 我们可以用两个角度 $\theta,\alpha$ 来表示:
\begin{equation*}
\begin{cases}
    x = \sin \theta \sin \alpha\\
    y = \sin \theta \cos \alpha\\
    z = \cos \theta
\end{cases}
\end{equation*}
以及柱面, 可以用角度和高度表示:
\begin{equation*}
\begin{cases}
    x = \sin \theta\\
    y = \cos \theta\\
    z = h
\end{cases}
\end{equation*}
(以上皆为个人yy, 毕竟找不到什么好的参考材料).\\
另外, 或许你熟悉子空间的维度的定义, 因此需要给出切空间的说明, 然后你会发现, 曲面上某点的切空间的维度就是曲面的维度.
\subsection{切空间}
\noindent
\begin{definition}
    设向量 $\bm{x} \in \mathbb{R}^n,\; V \subset \mathbb{R}^n$是一个线性子空间, 如果 $\forall \bm{y} \in V$, 都有
\begin{equation*}
    \bm{x}^\top \bm{y} = 0.
\end{equation*}
则称向量 $\bm{x}$ 正交于子空间 $V$.
\end{definition}
\noindent
假设用坐标表示的曲面为 $h(\bm{x}) = 0,\bm{x} \in \mathbb{R}^n$, $t \mapsto \bm{x}(t)$ 为曲面上的一条曲线(一维曲面). 因此曲线可以写成 $h(x_1(t),\cdots,x_n(t)) = 0$. 两边对 $t$ 求导:
\begin{equation*}
    h^{'}(t) = \nabla h(\bm{x})^\top \bm{x}^{'}(t) = 0.
\end{equation*}
即, 切线 $\bm{x}^{'}(t)$ 正交于曲面 $h(\bm{x})=0$.\\
我们把点 $\bm{x}$ 上所有的曲面的曲线的切线的集合称之为切空间$T_{\bm{x}}M$.\\
在有约束的最优化问题中, 对于判断一个KKT点是否为极值点, 我们通常需要用到上述切空间的定义:
\begin{equation*}
    T(\bm{x}) = \left\{\bm{y}\in\mathbb{R}^n|\nabla h(\bm{x})^\top \bm{y} = 0\right\}.
\end{equation*}
\end{document}